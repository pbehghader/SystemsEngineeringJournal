\section{Conclusions}
\label{sec:conclusion}
The increasing complexity of software-intensive systems and the rapid pace of change in technology are driving organizations' software total ownership costs more and more toward software maintenance. Examples are more, larger, and more complex software systems such as Internets of things and self-driving vehicles; increasing needs for software dependability and interoperability; increasing software autonomy; increasing data capture and data analytics; increasing legacy software, and mounting technical debt.

Fortunately, increasing data capture and data analytics capabilities are  providing organizations with stronger ways to analyze and reduce their software's technical debt. Commercial capabilities such as CAST, and technical debt analysis tools such as those summarized in Section \ref{sec:squaad} of this paper are just the beginning of the capabilities possible in this area.

However, much greater savings can be achieved by addressing three non-technical sources of technical debt due to overemphasis  on initial acquisition cost-effectiveness. These include shortfalls  in Life Cycle Management aspects; Personnel capabilities and participation aspects, and Maintenance methods, processes, and tools aspects. Drawing on the successful use of the Technology Readiness Level, Manufacturing Readiness Level, and System Readiness Level frameworks, this paper provides a similar Software Maintenance Readiness Framework (SMRF), based primarily on cumulative improvement of the three acquisition shortfalls that result in increased maintenance costs, that can enable development project management to anticipate and prepare for much more cost-effective software maintenance.