\section{Introduction}
\label{sec:introudction}

The inexorable increase in software demand and the pace of changes in software technology, competition, interdependencies, and sources of vulnerability, will seriously strain the capacity of the available software maintenance labor force.

We have been involved in several aspects of analyzing and addressing the root causes of expensive software maintenance.
These include serving on assessments of
exemplary and
problem software projects in government and industry;
evolving and calibrating models for estimating software development and maintenance costs;
leading a multi-year, multi-university US Department of Defense Systems Engineering Research Center to research and develop Software/Systems Qualities Ontology, Tradespace, and Affordability (SQOTA) capabilities;
and researching, developing, and evaluating promising methods, processes, and tools (MPTs) for improving software life cycle productivity and qualities.

We try to quantify such relationships where possible.
For example, the 2015 US General Accountability Office report \citep{dodaro2015government} identified annual US Government Information Technology (IT) expenditures of \$79 billion, of which \$58 billion were in operations and maintenance (O\&M).
Table \ref{tab:pdlcc} provides more detail on fractions of hardware and software costs from the 2008 Redman \citep{redman2008weapon} and  2009 Koskinen \citep{koskinen2009software} studies.


\begin{table}[htbp]
	\centering
	\caption{Percentage of Post-Deployment Life Cycle Cost}
	\resizebox{\columnwidth}{!}{
		
		% Table generated by Excel2LaTeX from sheet 'Sheet1'
		\begin{tabular}{ll}
			\toprule
			\textbf{Hardware (Redman, 2008)} & \textbf{Software (Koskinen, 2009)} \\
			\midrule
			12\% -- Missiles (average) & 75-90\% -- Business, Command-Control \\
			60\% -- Ships (average) & 50-80\% -- Complex cyber-physical systems \\
			78\% -- Aircraft (F-16) & 10-30\% -- Simple embedded software \\
			84\% -- Ground vehicles (Bradley) &  \\
			\bottomrule
		\end{tabular}%		
	}
	\label{tab:pdlcc}
	\vspace{-0.3cm}
\end{table}%

The SQOTA ontology identified Maintainability as not only supporting Affordability in terms of total ownership costs, but also supporting Changeability and Dependability (see Table \ref{tab:SERC_Stakeholder}). Maintainability supports Changeability in terms of rapid adaptability to new opportunities and threats, and also supports Dependability in terms of Availability, in that reducing Mean Time to Repair (MTTR) for a system with a given Reliability in terms of Mean Time Between Failures (MTBF) improves Availability via the equation Availability = MTBF / (MTBF+MTTR) \citep{IIS2:IIS2278}.

Note also that the SQOTA ontology’s definition of the key quality of Resilience or the combination of Dependability and Changeability is consistent with the INCOSE Systems Engineering Handbook’s definition of Resilience or “the ability to prepare and plan for, absorb or mitigate, recover from, or more successfully adapt to actual or potential adverse events”
\citep{incose2015systems,haimes2012systems}.

Many system development projects strongly focus on creating and evaluation the systems’ Initial Operational Capability (IOC). In doing so, they often miss opportunities to make the system more cost-effectively maintainable. 
More recently, both commercial and defense organizations have found it competitively critical to perform continuous development \& delivery, or DevOps, involving significant changes in preparing for post-IOC evolution, as recommended in the 2018 Design and Acquisition of Software for Defense Systems report \citep{DefenseScienceBoard}.

\begin{table}[htbp]
	\centering
	\caption{Upper Levels of SERC Stakeholder Value-Based System Quality (SQ) Means-Ends Hierarchy \citep{IIS2:IIS2278}}
	\resizebox{\columnwidth}{!}{
		
		% Table generated by Excel2LaTeX from sheet 'StakeHolder'
		\begin{tabular}{ L{2.7cm} | L{9cm}}
			\toprule
			Stakeholder Value-Based SQ Ends & Contributing SQ Means \\
			\midrule
			Mission Effectiveness & Stakeholders-Satisfactory Balance of Physical Capability, Cyber Capability, Human Usability, Speed, Endurability, Maneuverability, Accuracy, Impact, Scalability, Versatility, Interoperability, Domain-Specific Objectives \\
			\midrule
			Life Cycle Efficiency & Development and \textbf{Maintenance Cost}, Duration, Key Personnel, Other Scarce Resources; Manufacturability, Sustainability \\
			\midrule
			Dependability & Reliability, \textbf{Maintainability}, Availability, Survivability, Robustness, Graceful Degradation, Security, Safety \\
			\midrule
			Changeability & \textbf{Maintainability}, Modifiability, Repairability, Adaptability \\
			\midrule
			Composite Quality Attributes (QAs) &  \\
			\midrule
			Affordability & Mission Effectiveness, Life Cycle Efficiency \\
			\midrule
			Resilience & Dependability, Changeability \\
			\bottomrule
		\end{tabular}%
	}
	\label{tab:SERC_Stakeholder}%
	\vspace{-0.3cm}
\end{table}%


Section \ref{sec:opportunity_tree} summarizes the Opportunity Tree of ways to improve systems and software maintainability and life-cycle cost-effectiveness. 
Section \ref{sec:squaad} summarizes the Software Quality Understanding by Analysis of Abundant Data (SQUAAD) toolset for tracking a software project's incremental code commits, and analyzing and visualizing each commit's incremental and cumulative Technical Debt (TD).
Section \ref{sec:smrf} summarizes the Systems and Software Maintenance Readiness Framework (SMRF), that identifies needed maintenance readiness levels, and reflects that many sources of TD are non-technical.
Section \ref{sec:conclusion} summarizes the resulting conclusions.